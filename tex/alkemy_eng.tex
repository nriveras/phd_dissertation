\documentclass[a4paper,12pt]{article}

% Packages
\usepackage[utf8]{inputenc}
\usepackage[T1]{fontenc}
\usepackage{lmodern}
\usepackage{amsmath}
\usepackage{amssymb}
\usepackage{graphicx}
\usepackage{hyperref}
\usepackage{geometry}
\usepackage{fancyhdr}
\usepackage{setspace}
\usepackage{enumitem}
\usepackage{xcolor}
\usepackage{titlesec}
\usepackage{multicol}

% Page geometry
\geometry{margin=1in}

% Header and footer settings
\pagestyle{fancy}
\fancyhf{}
\fancyhead[L]{\leftmark}
\fancyfoot[C]{\thepage}

% Hyperlink setup
\hypersetup{
	colorlinks=true,
	linkcolor=blue,
	filecolor=magenta,      
	urlcolor=cyan,
}

% Section formatting
\titleformat{\section}{\normalfont\Large\bfseries}{\thesection}{1em}{}
\titleformat{\subsection}{\normalfont\large\bfseries}{\thesubsection}{1em}{}

% Document title
\title{Errata}
\author{}
\date{Version du 2 août 2024}

\begin{document}
	
	\maketitle
	\begin{multicols}{2}
		Les errata ci-ensuite sont officiels. Leur utilisation ne nécessite pas l’accord de votre adversaire, y compris en tournoi. Lorsqu’une réédition de la carte incorporant les errata a eu lieu, cela est précisé par une annotation du type \textbf{R 07/23}, indiquant la date de réédition. Ces cartes sont alors disponibles à la vente et gratuitement via Architekt. Les modifications récentes sont mises en évidence par une couleur.
		
		\section*{Livre de règles v2}
		Les numéros de page indiqués correspondent à la version "couverture rigide" du livre de règles.
		
		\subsection*{Notions élémentaires (suite) (p. 24)}
		\begin{itemize}
			\item Ajouter
			\begin{itemize}
				\item "Pions, marqueurs et jetons : lors d’une partie d’Alkemy, les joueurs peuvent être amenés à faire usage de ces éléments, définis comme suit :
				\begin{itemize}
					\item un pion est un élément de jeu d’un pouce de diamètre et taille 0, sauf précision contraire. De tels éléments peuvent être traversés par les figurines durant leur mouvement. Ils ne peuvent pas être empilés et une figurine ne peut pas terminer son mouvement même partiellement sur un pion;
					\item un marqueur sert à indiquer des états, attributs d’une figurine ou des éléments spécifiques au scénario. Ce n’est un élément de jeu et il n’a aucun impact sur les mouvements des figurines;
					\item un jeton est un élément de jeu de diamètre 0.5p et taille 0 qui peut être traversé par les figurines durant leur mouvement."
				\end{itemize}
				\item "Gabarits : sauf mention contraire explicite, un gabarit est un élément de jeu pouvant être traversé librement, placé sur tout élément de jeu (y compris un autre gabarit ou une figurine). Un tel élément n’a pas de taille et n’influe pas sur les LdV."
			\end{itemize}
		\end{itemize}
		
		\subsection*{Faire agir ses figurines, généralités (p. 30)}
		\begin{itemize}
			\item Ajouter
			\begin{itemize}
				\item "Actions simultanées : dans le cas où deux actions ou compétences doivent être résolues simultanément, procéder comme suit :
				\begin{itemize}
					\item si la résolution des deux actions ou compétences incombe au même joueur, ce dernier décide de l’ordre effectif de résolution;
					\item si les deux joueurs sont concernés, le dernier joueur n’ayant pas eu l’initiative prend la décision;
					\item si les deux joueurs sont concernés et que les actions sont à résoudre avant le premier tour, le joueur qui a terminé son déploiement en premier prend la décision."
				\end{itemize}
				\item "Lorsqu’une compétence nécessite de dépenser des ressources et que sa résolution est soumise à conditions (portée, LdV, . . .), les ressources sont dépensées avant de déterminer si la compétence peut être résolue ou non. Dans le cas d’une compétence utilisable un nombre de fois limité par tour, elle est considérée comme ayant été utilisée une fois, qu’elle puisse être résolue ou non."
			\end{itemize}
		\end{itemize}
		
		\subsection*{Alchimie (p. 34)}
		\begin{itemize}
			\item Ajouter
			\begin{itemize}
				\item "Jouer la Loge : les 4 composants alchimiques sont laissés au choix du joueur. Il est possible de mélanger les affinités (par exemple de jouer un composant de chaque élément). En tournoi, ce choix s’applique sans modification possible pour la durée de l’événement."
			\end{itemize}
		\end{itemize}
		
		\subsection*{Affinité Eau (p. 35)}
		\begin{itemize}
			\item Ajouter "L’augmentation de portée des formules ne s’applique pas aux formules (par exemple à aire d’effet) déjà lancées au moment de la collecte du composant."
		\end{itemize}
		
		\section*{Cartes objectifs secrets}
		\begin{itemize}
			\item Tactique batracienne : remplacer par “Ne jouer que des CC de Bottes ou Inactif durant un tour.” Cette carte rapporte 2PV (et non 4) + 1PV en cas de victoire.
		\end{itemize}
		
		\section*{Recueil de scénarios 1}
		\begin{itemize}
			\item Conquête territoriale (p. 9) : paragraphe Détruire un drapeau, remplacer "cible initiale d’une charge" par "cible initiale d’une charge réussie".
		\end{itemize}
		
		\section*{Recueil de scénarios 2}
		\begin{itemize}
			\item Objets maudits (p. 15) : paragraphe Transmettre un objet maudit à une figurine amie, remplacer "Une figurine active peut prendre ou transmettre un objet à une figurine amie" par "Une figurine active peut prendre ou transmettre un objet à une figurine amie au contact".
		\end{itemize}
		
		\section*{Recueil de scénarios 3}
		\begin{itemize}
			\item Objets maudits (p. 15) : paragraphe Transmettre un objet maudit à une figurine amie, remplacer "Une figurine active peut prendre ou transmettre un objet à une figurine amie" par "Une figurine active peut prendre ou transmettre un objet à une figurine amie au contact".
		\end{itemize}
		
		\section*{Recueil de scénarios 4}
		\begin{itemize}
			\item Échange d’otages (p. 9) :
			\begin{itemize}
				\item Paragraphe Conditions de victoire, remplacer "9 PS" par "12 PS".
				\item Paragraphe Action ligoter un otage, remplacer par "Une figurine au contact d’un otage adverse en état grave ou critique et ayant le statut libre peut dépenser 1 PA pour que cet otage obtienne le statut Prisonnier. Une figurine à moins de 3p d’un otage adverse en état critique et ayant le statut Libre, peut dépenser 2PA pour effectuer une marche et se mettre au contact de cet otage qui obtient alors le statut Prisonnier."
				\item Paragraphe Valeur des otages, remplacer "Les joueurs marquent 3 PS à chaque fois qu’un otage ami est retiré du jeu." par "Les joueurs marquent 4 PS à chaque fois qu’un otage ami est retiré du jeu."
			\end{itemize}
			\item Étranges infectés (p. 11) : paragraphe Infectés, remplacer "À la fin de chaque tour, chaque figurine amie au contact d’au moins un infecté rapporte 1 PS." par "À la fin de chaque tour, chaque infecté au contact d’au moins une figurine amie rapporte 1 PS." et "Lorsqu’un infecté est pris pour cible" par "Lorsqu’un infecté est pris pour cible par une action qui serait réussie en l’absence de mouvement de ce dernier".
		\end{itemize}
		
		\section*{Compétences transverses}
		\begin{itemize}
			\item Hémotoxique : remplacer par "Si cette figurine est en état Grave (jaune) ou Critique (rouge), chaque fois que la figurine subit des DOM provenant d’une attaque portée par une figurine En Combat et au contact socle à socle avec elle, cette dernière subit 1 DOM en retour."
			\item Armes empoisonnées : ajouter "Les dés malus ainsi infligés sont cumulables,
