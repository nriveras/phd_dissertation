\chapter*{Abstract}
\markboth{Abstract}{Abstract}

Biological soil crusts (biocrusts) and climate effects profoundly shape soil stabilization, erosion, and nutrient dynamics, especially across diverse environmental gradients like the Chilean Coastal Range (arid to humid). This research explored these intricate interactions by integrating field observations, rainfall simulations, laboratory experiments, and advanced analytical techniques. The focus was on the interplay between biocrusts, microbial communities, plant roots, and fundamental soil properties.

Key findings reveal that biocrusts significantly enhance soil aggregate stability, particularly in drier regions, thereby reducing surface runoff and erosion. Their stabilizing influence, however, lessens in humid climates where dense vegetation becomes the dominant factor. Biocrusts modify hydrological pathways, typically decreasing surface flow while sometimes increasing percolation. They also modulate carbon (C) and nitrogen (N) fluxes in climate-dependent ways, influencing both sediment-bound and dissolved nutrient transport.

The critical role of microbial communities in soil aggregation and development was confirmed, with their activity and resilience strongly linked to climate legacy and moisture patterns (e.g., wetting-drying cycles). Plant roots emerged as powerful drivers of macroaggregation, exerting distinct influences during their living (rhizosphere) and decaying (detritusphere) phases, which in turn affects microbial succession and organic matter protection. Overall, this work highlights the interconnected, context-dependent nature of these biotic and abiotic factors in governing soil structure and function.

