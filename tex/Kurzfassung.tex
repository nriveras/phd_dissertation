\chapter*{Kurzfassung}
\markboth{Kurzfassung}{Kurzfassung}

\begin{justify}
Biologische Bodenkrusten (Biokrusten) und Klimaeffekte prägen maßgeblich die Bodenstabilisierung, Erosion und Nährstoffdynamik, insbesondere entlang diverser Umweltgradienten wie der chilenischen Küstenkordillere (arid bis humid). Diese Forschung untersuchte diese komplexen Wechselwirkungen durch die Integration von Feldbeobachtungen, Regensimulationen, Laborexperimenten und fortschrittlichen Analysetechniken. Der Fokus lag auf dem Zusammenspiel zwischen Biokrusten, mikrobiellen Gemeinschaften, Pflanzenwurzeln und grundlegenden Bodeneigenschaften.

Wichtige Ergebnisse zeigen, dass Biokrusten die Aggregatstabilität des Bodens signifikant erhöhen, besonders in trockeneren Regionen, wodurch Oberflächenabfluss und Erosion reduziert werden. Ihr stabilisierender Einfluss nimmt jedoch in humiden Klimazonen ab, wo dichte Vegetation zum dominierenden Faktor wird. Biokrusten modifizieren hydrologische Pfade, indem sie typischerweise den Oberflächenabfluss verringern, während sie manchmal die Perkolation erhöhen. Sie modulieren auch Kohlenstoff- (C) und Stickstoff- (N) Flüsse auf klimaabhängige Weise und beeinflussen sowohl den sedimentgebundenen als auch den gelösten Nährstofftransport.

Die entscheidende Rolle mikrobieller Gemeinschaften bei der Bodenaggregation und -entwicklung wurde bestätigt, wobei ihre Aktivität und Resilienz stark mit Klima-Legacy-Effekten und Feuchtemustern (z.B. Benetzungs-Trocknungs-Zyklen) verknüpft sind. Pflanzenwurzeln erwiesen sich als starke Treiber der Makroaggregation, die während ihrer lebenden (Rhizosphäre) und zerfallenden (Detritussphäre) Phasen unterschiedliche Einflüsse ausüben, was wiederum die mikrobielle Sukzession und den Schutz organischer Substanz beeinflusst. Insgesamt hebt diese Arbeit die vernetzte, kontextabhängige Natur dieser biotischen und abiotischen Faktoren bei der Steuerung von Bodenstruktur und -funktion hervor.
\end{justify}