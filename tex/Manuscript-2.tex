\chapter{Biocrusts as climate-dependent regulators of erosion, water and nutrient cycling}
\label{chap:manuscript2} % Label for potential cross-referencing (start page)

\section*{Abstract} % Use \section* for unnumbered sections if needed

Biocrusts, complex communities of organisms, significantly alter surface and sub-surface processes, including water infiltration, and the cycling of carbon (C) and nitrogen (N). While their importance is recognized, studies across broad climatic gradients are scarce. We conducted rainfall simulation experiments at four sites along a 910 km Chilean Coastal Range transect, representing coastal and inland semi-arid, mediterranean, and humid climates. We quantified overland flow, sediment discharge, and fluxes of C and N in percolating water, comparing paired biocrust-covered and bare soil surfaces. Our findings reveal that biocrusts, compared to bare soil, significantly increased cumulative infiltration rates across all climates, indicating enhanced saturated hydraulic conductivity. Biocrust presence reduced runoff by 73\% in humid climates and sediment flux by 80\% in inland semi-arid climates. On average, soil erosion was reduced by up to 69\%. Additionally, biocrusts significantly reduced C loss via erosion. Dissolved organic carbon and nitrogen fluxes were also strongly influenced by the presence of biocrusts. Overall, our study demonstrates that, though the presence of biocrust significantly reduced erosion, water and nutrient dynamic are strongly influence by its presence and the climates, showing that biocrusts act as climate-dependent regulators of crucial surface and sub-surface transport processes.

\section*{Keywords}

Biocrusts, Erosion, Infiltration, Climate gradient, Carbon, Nitrogen, Runoff

\section{Introduction}

Biological soil crusts (BSC) are formed by a close association between soil particles and various proportions of photoautotrophic organisms, such as cyanobacteria, algae, lichens, and bryophytes, along with heterotrophic organisms like bacteria, fungi, and archaea that live within or immediately above the top millimeters of the soil (Weber et al., 2022). They are a primary soil cover in arid environments, where the scarce availability of water limits the establishment of higher plants (Ding and Eldridge, 2020; Weber et al., 2022). However, biocrusts are also present in mesic environments without water limitation, where vascular plants and litter reduce their ability to compete (Corbin and Thiet, 2020; Gall et al., 2022b). Nevertheless, biocrusts can be abundant in such areas when disturbance and the subsequent ecological succession provide favorable conditions for their establishment (Büdel and Colesie, 2014; Gall et al., 2022a).

Biocrust composition, growth, and survival depend on climatic factors such as temperature, moisture availability, and rainfall characteristics (Belnap, 2003). Moreover, climatic diversity shapes the morphological, chemical and physiological characteristics of the different biocrusts (Concostrina-Zubiri et al., 2014). In arid environments, biocrusts typically dominated by algae, cyanobacteria, and lichens, form smooth and flat surfaces that concentrate a large amount of microbial biomass and serve as nutrient sources within these ecosystems (Weber et al., 2022). In these settings, cyanobacteria can fix large amounts of nitrogen and produce reserve exopolysaccharides (EPS) that support other organisms (Rodríguez‐Caballero et al., 2018; Samolov et al., 2020).

In addition, under humid conditions, biocrusts have a higher proportion of fungi and bryophytes, resulting in rough surfaces with a remarkable capacity to store capillary water, increased pore space in their structure and enhanced carbon fixation (Riveras-Muñoz et al., 2022; Weber et al., 2022). These changes in surface roughness affect the surface hydrology of the soil by modifying the residence time of water on the surface (Kidron et al., 2022). Changes in water residence time, in turn, affect the distribution of infiltration and runoff, alter surface redox conditions, modify soil erodibility, and regulate the release of solutes into water (Kalnejais et al., 2010). This emphasizes the ability of biocrusts to modify water and sediment fluxes. In particular, lichen-dominated dryland biocrusts can enhance surface runoff and reduce sediment yield by surface clogging and subsequently surface saturation, and runoff initiation (Kidron et al., 2021). They can also dramatically alter albedo, soil temperature dynamics and thus consequently the potential evapotranspiration (Liu and She, 2020; Rutherford et al., 2017; Whitney et al., 2017; Xiao et al., 2019). Moss-dominated biocrusts with a rough morphology have a high water-holding potential and can decrease both runoff and sediment yield (Juan et al., 2023; Seitz et al., 2017; Silva et al., 2019; Zeng et al., 2025). Other effects of biocrusts that also include near-surface infiltration within the biocrust cover and its effect on percolation report enhanced infiltration and percolation in semi-arid ecosystems (Chamizo et al., 2016) and reclaimed soils (Gypser et al., 2016), alteration rainwater flow in drylands (Li et al., 2022), and reduced infiltration while increasing erosion resistance in disturbed forests (Szyja et al., 2023). Moreover, comparative experimental studies focusing on soil water movement through different types of biocrust remain limited.

Along with their ability to modify water and sediment fluxes, biocrust communities significantly shape carbon (C) and nitrogen (N) cycles. Many researchers point biocrusts as one of the most important factors that initially influence soil organic carbon (SOC) in the uppermost soil horizon before higher plants appear (Belnap et al., 2007). Furthermore, Young et al. (2022) demonstrated that biocrusts facilitate the vertical movement of soluble carbon and nutrients from the surface to subsurface mineral soils, thereby enhancing overall carbon sequestration. The significance of biocrusts in the carbon cycle extends beyond their local role in building SOC (Witzgall et al., 2024) , with these communities estimated to account for 7\% of total net carbon uptake and 50\% of terrestrial nitrogen fixation (Elbert et al., 2012). They further enhance C fixation through photosynthetic sequestration (Belnap et al., 2016; Grote et al., 2010). An increase in soil aggregate stability hinders C discharge by erosion (Riveras-Muñoz et al., 2022; Xiao et al., 2022), enhances soil fertility (Kheirfam et al., 2017), and fosters the establishment of other vascular and non-vascular organisms, further increasing C storage potential (Molina-Montenegro et al., 2016). In disturbed temperate forests, Gall et al. (2024) found that mosses significantly influence the relocation of SOC and total N via soil erosion and percolation, further highlighting biocrusts’ impact on nutrient redistribution. Additionally, biocrusts play an active role in the N cycle as they are responsible for approximately 40 to 85\% of N fixation worldwide, mainly through the activity of cyanobacteria , enhances soil fertility (Kheirfam et al., 2017), and foster the establishment of other vascular and non-vascular organisms, further increasing C storage potential (Molina-Montenegro et al., 2016). Additionally, biocrusts play an active role in the N cycle as they are responsible for approximately 40 to 85\% of N fixation worldwide, mainly through the activity of cyanobacteria (Rodríguez‐Caballero et al., 2018; Samolov et al., 2020). As for C, biocrusts immobilize N by incorporating it into their biomass, thereby reducing losses by leaching or volatilization (Nevins et al., 2020; Pan et al., 2021). Furthermore, biocrusts can mineralize N from SOC, making it available to other organisms in the soil ecosystem (Weber et al., 2015).

All these properties of biocrusts impact soil erosion, a central process that shapes the Earth’s surface (Luetzenburg et al., 2020; Scholten and Seitz, 2019). Water-induced erosion occurs in humid and semi-humid regions (Gholzom and Gholami, 2012; Khaleghi and Varvani, 2018) but also occurs in arid environments due to rare extreme rainfall events (Hu et al., 2022). Biocrusts not only influence infiltration and overland flow, but also form a physical barrier against erosive agents, partially absorbing the kinetic energy of running water and falling raindrops. A reduction of runoff by 25.6\% and sediment discharge by 75.5\% was observed when comparing runoff plots with biocrust cover below 10\% and above 50\% in early successional subtropical forests (Seitz et al., 2017). Similar effects of biocrusts on soil erosion have been reported in arid (Bowker et al., 2018; Eldridge et al., 2021), temperate (Gall et al., 2022a) and humid environments (Guo et al., 2022; Zhao et al., 2014). Another process of land surface stabilisation by biocrusts is the formation of aggregates from organic and mineral particles through the secretion of bacterial metabolites such as exo- and lipopolysaccharides (Costa et al., 2018; Tourney and Ngwenya, 2014; Xiao et al., 2022). Further, the trapping of soil particles within the structures of biocrusts helps prevent soil erosion (Riveras-Muñoz et al., 2022; Rodriguez et al., 2024; Xiao et al., 2022).

Biocrusts are increasingly observed and described outside their classical dryland habitats—hyper-arid, arid, semi-arid, and dry sub-humid habitats (Gall et al., 2022b; Weber et al., 2022). Furthermore, investigations into climate variability have covered multiple scales. For example, Munoz-Martin et al. (2019) studied cyanobacterial biocrust diversity along an aridity gradient in Mediterranean semi-arid soils and found that temperature and precipitation determine biocrust composition, with a greater prevalence of extremotolerant in harsher climates. Regarding topography, Castillo-Monroy et al. (2016) reported that species composition and richness of biocrusts increase with elevation in tropical shrublands. Additionally, Ding and Eldridge (2020) observed that, on smaller spatial scales in Australia, microsite differences correlate with an increase in biocrust cover as aridity rises, while Riveras-Muñoz et al. (2022) demonstrated that soil aggregate stabilization by biocrust improved from arid to humid climates in Chile. Both studies reveal that the dominant structuring mechanisms of biocrusts shift with climate: under dry conditions, biocrusts stabilize the soil surface through exopolysaccharide production that promotes aggregate formation and structural stability, evidenced by the development of water-stable aggregates; whereas under humid conditions, biocrust structures further entangle and stabilize the soil surface, resulting in a more defined crust (Riveras-Muñoz et al., 2022). Although studies investigating the role of biocrusts in regulating water, sediment, and C and N fluxes in different climates are limited, the interactions and feedback mechanisms between the underlying processes are not yet fully understood.

To enhance our understanding of the interrelation between climate and the multifunctional roles of biocrusts in regulating water, sediment, and matter fluxes at the soil surface and within the topsoil, we conducted a field experiment comparing land surfaces with and without biocrusts. We quantified changes in C and N fluxes, both in particulate and water-dissolved forms influenced by the presence of biocrusts. Additionally, we examined how biocrusts modulated the interactions between water, sediment, and nutrient fluxes along a gradient ranging from arid to humid climates. We introduced a new measuring device capable of simultaneously sampling runoff, sediment and seepage flow in an undisturbed soil monolith during simulated rainfall events. This study focused on four sites within the Coastal Mountain Range of Chile, representing a climate gradient that includes coastal and inland semi-arid, Mediterranean and humid climates. All study sites (Figure 1) have a comparable topography and almost the same parent materials. Our experimental approach utilizing undisturbed soil monoliths subjected to standardized rainfall simulations allows us to assess biocrusts' effects on various fluxes. Our main hypotheses are that (i) biocrusts modify runoff, sediment discharge and percolation flow, including liquid and solid C and N fluxes and reduce soil erosion irrespective of climatic conditions, but (ii) feedbacks between overland flow, percolation flow, sediment and liquid and solid C and N fluxes are climate specific

\section{Materials and methods}
\subsection{Study sites}
\subsection{Rainfall simulation experiment}

At the research sites, one-square-meter plots were established for the actual rainfall simulation experiments. Each plot was located at top slope with a south-facing orientation. The setups considered the presence of site-typical biocrust communities, similar slope and aspect, and a lack of anthropogenic disturbance and ensured that the distance between each plot did not exceed 30 meters. Rainfall simulation was designed as a factorial, completely randomized experiment with eight treatments (four sites, each with and without biocrust). Five field replicates and three soil samples as technical replicates were taken, resulting in a sample size of n = 120 rainfall simulations.

Infiltration boxes (Figure 3) were developed as part of a rainfall simulation experiment to measure runoff and percolation flow, including their matter content. Undisturbed soil samples were collected using cutting frames (20 cm × 30 cm × 7 cm, Figure 3a), carefully installed with minimal surface and subsurface disturbances (Figure 3c, Seitz (2015)). Cutting frames are made from 1 mm metal plates, sharpened in the bottom sides, and include an upper border to transfer the strength to them without direct contact with the soil and a lower one to stop the frame from burying beyond the desired height of the sample (Figure 3a). Subsequently, the cutting frames were excavated around, and a metal plate was inserted underneath it (Figure 3d). The cutting frames with the soil samples were covered with metal plates, wrapped in plastic foil, and carefully transported to a flat area with water available. Then, the wrappings were removed, and the frames with the soil samples were stacked on a permeable metal plate and placed inside the soil erosion flux box (Figure 3a) designed as steel containers with a triangular surface runoff gutter and an outlet at the bottom to capture the percolation flow (Figure 3a). Soil depth within the boxes is 7 cm. In the case of the presence of small plants, they were cut flush with the surface using a scissor and paying attention to not pull them and avoid surface disturbances. The water content of each sample was measured by a TDR probe (Delta-T Devices Ltd. Cambridge, UK) when sampling, using the average of 3 measurements directly next to the sampling area. Perpendicular photographs were taken on each sample with a digital camera (Sony ILCE-6000 equipped with a lens SELP1650, Tokyo, Japan) and processed with the grid quadrat method overlying a digital grid of 100 subdivisions and separating biocrusts by visual inspection (Belnap et al., 2001) to assess the biocrust ground cover.

Rainfall simulations were conducted near the sampling site with the Tübingen rainfall simulator (Iserloh et al., 2013; Seitz, 2015), equipped with a Lechler 460.788.30 nozzle and adjusted to a falling height of 3.5 m. The stack with the sample was placed inside the rainfall simulator and set to a 10° slope (Figure 3e). A rainfall event was simulated using an intensity of 45 mm h⁻¹ sustained over a 30‐minute period. According to regional intensity-duration-frequency analyses for central Chile (Pizarro-Tapia et al., 2020), such an intensity falls within the extreme rainfall category, well above the heavy precipitation threshold even for relatively wet climates. This extreme intensity was selected to exceed the soil infiltration capacities and reliably generate surface runoff at all study sites. In particular, the four sites represent a pronounced climatic gradient, ensuring that the simulated storm event is sufficiently intense to produce runoff under the diverse hydrological conditions encountered across these regions. The time to start runoff and percolation generation was recorded with a timer. Sediment-water samples were collected in bottles at the runoff gutter and percolation valve. The volume of water was registered with a graduated beaker. The samples were left to sedimentation by gravity, and a water sample was extracted from the supernatant using a siphon and frozen at −4°C. The remaining sample was dried in an oven at 105°C until no water was observed and then for 48 hours. The weight of the sediment was measured using a balance, and the total C and N content of the sediment was subsequently analyzed. The sediment load was calculated by dividing the sediment amount by the water volume. The nutrient load on sediments was measured using an elemental analyzer (Vario EL III, Elementar Analysensysteme GmbH, Hanau, Germany). The water samples were filtered to 0.45 µm and analyzed for DOC and DON using a Multi N/C 2100 S from Analytik Jena (Jena, Germany).

\subsection{Statistical analysis}

The experiment was designed as a factorial combination of site (4 levels) and biocrust (2 levels). Differences in the properties analyzed were proven by ANCOVA, using the soil baseline variables clay, silt, sand, CT, and NT as covariates. When covariates were not significant, analyzes continued using ANOVA. ANCOVA and ANOVA were implemented in R using the package stats 4.2.1 (R Core Team, 2022). In case normality or homoscedasticity were not accomplished, automatic data transformation was applied using the package bestNormalize 1.8.3 (Peterson, 2021; Peterson and Cavanaugh, 2020). We evaluated the normal distribution of the data using Shapiro-Wilk's Normality Test (p> 0.05) implemented in R stats 4.2.1 (R Core Team, 2022) and homoscedasticity by Levene's Test (p<0.05) implemented in the car 3.1-0 package (Fox and Weisberg, 2018). The individual significance of treatments was assessed by the Dunn–Šidák correction implemented in the package multcomp 1.4-20 (Hothorn et al., 2008).

\section{Results}
\subsection{Surface runoff, percolating water fluxes, and their interrelation with biocrust cover along different climatic conditions}

Our findings reveal that biocrusts play a critical role in modulating associated with erosion, runoff, and percolation across varying climatic conditions (Table 1 and 2). Overall, the time required to start runoff was significantly longer in NA, with the shortest time observed in the coastal arid site, SG. Biocrusts significantly delayed runoff initiation by 97.7\%, mainly due to increased surface roughness. This delay was more pronounced in NA, where the combination of enhanced roughness and high infiltration capacity extended the time for runoff initiation by three to four times. However, these effects did not extend to the subsoil, as biocrust had no measurable impact on the time needed for percolation to start.

The total runoff volume was significantly higher in SG compared to the other sites. Biocrust contributed to a significant reduction of the runoff volume, averaging a 28.0\% reduction, corresponding to a significant 50.0\% reduction in percolation. Regarding site-specific effects, biocrusts significantly reduced the runoff volume by 72.4\% in NA. Contrarily, a contrasting trend in LC was observed, with a 36.4\% increase in runoff volume associated with the biocrust presence.

The climatic gradient significantly influenced the total sediment transported by runoff, decreasing sediment mobilization as humidity increased. Biocrusts were pivotal in reducing sediment transport via runoff, resulting in an overall decrease of 69.9\%. However, this reduction in runoff-associated sediment transport was accompanied by a 28.3\% increase in the sediment mobilization via percolation. The interaction between the study site and biocrust presence showed significant sediment reductions across the northern sites, with NA exhibiting a similar trend, albeit without statistical significance.

The sediment load in the runoff, measured as sediment concentration in water, was significantly reduced by an average of 60.9\% in the presence of biocrust. In contrast, sediment concentration in percolation increased significantly by 58.3\%, highlighting a shift in sediment dynamics influenced by biocrust interactions across the study sites.

  % --- Table 1 Code (Runoff Fluxes) ---
\begin{table}[htbp] % Changed from sidewaystable
    \centering
    \footnotesize % Reduced font size for the table
    \sisetup{separate-uncertainty = true, table-align-uncertainty = true}
    \begin{threeparttable}
      \caption{Surface runoff fluxes on the study sites (SG: Santa Gracia, QdT: Quebrada de Talca, LC: La Campana, NA: Nahuelbuta) with (+) and without (-) biocrust (BSC) cover. Values correspond to mean $\pm$ standard deviation (SD) of five field replicates. A letter-based display of Šidák correction accompanies surface runoff parameters\tnote{a}, and the significance of the studied factors is shown as a p-value and list of covariates (C\textsubscript{T}, N\textsubscript{T}: total carbon and nitrogen content). Different letters indicate statistically significant different values based on our data, ns: non-significant.}
      \label{tab:runoff_fluxes}
      \setlength{\tabcolsep}{3pt} % Reduced further slightly
  
      \begin{tabular}{@{} l l S[table-format=3.1, table-figures-uncertainty=4] @{\,} c
                             S[table-format=2.0, table-figures-uncertainty=2] @{\,} c
                             S[table-format=3.0, table-figures-uncertainty=3] @{\,} c
                             S[table-format=2.1, table-figures-uncertainty=3] @{\,} c
                          @{}}
        \toprule
        \multicolumn{2}{@{}l}{\multirow{3}{*}{\textbf{Factor}}} &
        {\makecell{\textbf{Time to}\\\textbf{start}\\\textbf{runoff}}} & & % Added line break for width
        {\makecell{\textbf{Runoff}}} & &
        {\makecell{\textbf{Sediment}\\\textbf{in runoff}}} & & % Added line break
        {\makecell{\textbf{Sediment}\\\textbf{load of}\\\textbf{runoff}}} & \\ % Added line breaks
        \cmidrule(lr){3-3} \cmidrule(lr){5-5} \cmidrule(lr){7-7} \cmidrule(lr){9-9}
        \multicolumn{2}{@{}l}{} & {[\si{\second}]} & & {[\si{L.h^{-1}}]} & & {[\si{g.m^{-2}.h^{-1}}]} & & {[\si{g.L^{-1}.m^{-2}}]} & \\
        \midrule
        \multicolumn{2}{@{}l}{\textbf{Mean} $\pm$ \textbf{SD}} & & & & & & & & \\
        & Site \quad SG  & 65.1 \pm 20.7 & {(a)}  & 49 \pm 18 & {(b)}  & 617 \pm 473 & {(c)}  & 12.1 \pm 7.9 & {(a)} \\
        & \phantom{Site \quad} QdT & 78.7 \pm 30.4 & {(a)}  & 40 \pm 16 & {(a)}  & 398 \pm 459 & {(b)}  & 9.6 \pm 9.6 & {(a)} \\
        & \phantom{Site \quad} LC  & 83.5 \pm 56.0 & {(a)}  & 39 \pm 22 & {(a)}  & 241 \pm 293 & {(b)}  & 7.3 \pm 8.8 & {(a)} \\
        & \phantom{Site \quad} NA  & 236.7 \pm 273.0 & {(b)} & 44 \pm 43 & {(a)}  & 28 \pm 44 & {(a)}   & 3.0 \pm 14.0 & {(a)} \\
        \addlinespace
        & Biocrust \quad BSC+ & 154.0 \pm 211.5 & {(b)} & 36 \pm 21 & {(a)}  & 149 \pm 222 & {(a)}  & 4.5 \pm 10.4 & {(a)} \\
        & \phantom{Biocrust \quad} BSC-- & 77.9 \pm 33.8 & {(a)}  & 50 \pm 31 & {(b)}  & 495 \pm 492 & {(b)}  & 11.5 \pm 10.0 & {(b)} \\
        \midrule
        \multicolumn{2}{@{}l}{\textbf{Site*Biocrust}} & & & & & & & & \\
        & SG \quad BSC+  & 62.7 \pm 21.2 & {(ab)}  & 44 \pm 16 & {(ab)} & 340 \pm 340 & {(bc)} & 7.5 \pm 5.7 & {(abc)} \\
        & SG \quad BSC--  & 67.5 \pm 20.6 & {(ab)}  & 54 \pm 18 & {(ab)} & 873 \pm 456 & {(d)}  & 16.7 \pm 7.1 & {(de)} \\
        & QdT \quad BSC+ & 87.3 \pm 37.5 & {(ab)}  & 38 \pm 15 & {(ab)} & 131 \pm 96 & {(ab)}  & 3.3 \pm 2.0 & {(abd)} \\
        & QdT \quad BSC-- & 70.1 \pm 18.7 & {(ab)}  & 42 \pm 18 & {(ab)} & 665 \pm 524 & {(cd)} & 15.8 \pm 10.2 & {(ce)} \\
        & LC \quad BSC+  & 85.8 \pm 67.1 & {(ab)}  & 45 \pm 20 & {(ab)} & 87 \pm 94 & {(a)}   & 2.1 \pm 1.9 & {(a)} \\
        & LC \quad BSC--  & 81.2 \pm 44.6 & {(ab)}  & 33 \pm 23 & {(ab)} & 395 \pm 344 & {(bc)} & 12.6 \pm 9.9 & {(bcde)} \\
        & NA \quad BSC+  & 380.4 \pm 329.5 & {(b)} & 19 \pm 23 & {(a)}  & 16 \pm 49 & {(a)}   & 5.2 \pm 19.9 & {(abcde)} \\
        & NA \quad BSC--  & 93.0 \pm 40.2 & {(a)}   & 69 \pm 45 & {(b)}  & 40 \pm 36 & {(a)}   & 0.9 \pm 1.2 & {(abcde)} \\
        \midrule
        \multicolumn{2}{@{}l}{\textbf{Significance} ($p$-value)} & & & & & & & & \\
        & Site         & \num{1.03e-05} & \tnote{*} & \num{0.0304} & \tnote{*} & \num{2.21e-14} & \tnote{*} & \num{1.93e-09} & \tnote{*} \\
        & Biocrust     & \num{0.01452} & \tnote{*} & \num{0.0034} & \tnote{*} & \num{4.37e-11} & \tnote{*} & \num{7.57e-10} & \tnote{*} \\
        & Site*Biocrust& \num{0.0166}  & \tnote{*} & \num{1.83e-05} & \tnote{*} & \num{0.00129}  & \tnote{*} & \num{0.000177} & \tnote{*} \\
        \midrule
        \multicolumn{2}{@{}l}{\textbf{Covariates}} &
        \multicolumn{2}{@{}l}{clay + C\textsubscript{T}} &
        \multicolumn{2}{@{}l}{silt + sand + N\textsubscript{T}} &
        \multicolumn{2}{@{}l}{} &
        \multicolumn{2}{@{}l}{\makecell[tl]{clay + sand + C\textsubscript{T} + \\ soil water content}} \\
        \bottomrule
      \end{tabular}
      \begin{tablenotes}[para,flushleft]
        \item[a] Letters indicate significant differences between means according to post-hoc tests with Šidák correction for multiple comparisons. Levels not sharing any letter are significantly different (p < 0.05).
        \item[*] Statistically significant effect (p < 0.05).
        \item[SD] Standard Deviation.
        \item[C\textsubscript{T}] Total Carbon content.
        \item[N\textsubscript{T}] Total Nitrogen content.
      \end{tablenotes}
    \end{threeparttable}
  \end{table}
  
  % ... text between tables ...
  
  % --- Table 2 Code (Percolation Fluxes) ---
  \begin{table}[htbp] % Changed from sidewaystable
    \centering
    \footnotesize % Reduced font size for the table
    \sisetup{separate-uncertainty = true, table-align-uncertainty = true}
    \begin{threeparttable}
      \caption{Percolating water fluxes on the study sites (SG: Santa Gracia, QdT: Quebrada de Talca, LC: La Campana, NA: Nahuelbuta) with (+) and without (-) biocrust (BSC) cover. Values correspond to mean $\pm$ standard deviation (SD) of five field replicates. A letter-based display of Šidák correction accompanies surface runoff parameters\tnote{a}, and the significance of the studied factor is shown as a p-value and list of covariates (C\textsubscript{T}, N\textsubscript{T}: total carbon and nitrogen content). Different letters indicate statistically significant different values based on our data, ns: non-significant.}
      \label{tab:percolation_fluxes}
      \setlength{\tabcolsep}{3pt} % Reduced further slightly
  
      \begin{tabular}{@{} l l S[table-format=3.2, table-figures-uncertainty=4] @{\,} c
                             S[table-format=2.2, table-figures-uncertainty=3] @{\,} c
                             S[table-format=2.0, table-figures-uncertainty=2] @{\,} c
                             S[table-format=1.1, table-figures-uncertainty=3] @{\,} c
                          @{}}
        \toprule
        \multicolumn{2}{@{}l}{\multirow{3}{*}{\textbf{Factor}}} &
        {\makecell{\textbf{Time to start}\\\textbf{percolation}\\\textbf{flow}}} & & % Added line break
        {\makecell{\textbf{Percolation}}} & &
        {\makecell{\textbf{Sediments in}\\\textbf{percolation}\\\textbf{flow}}} & & % Added line breaks
        {\makecell{\textbf{Sediment load}\\\textbf{in percolation}}} & \\ % Added line break
        \cmidrule(lr){3-3} \cmidrule(lr){5-5} \cmidrule(lr){7-7} \cmidrule(lr){9-9}
        \multicolumn{2}{@{}l}{} & {[\si{\second}]} & & {[\si{L.h^{-1}}]} & & {[\si{g.m^{-2}.h^{-1}}]} & & {[\si{g.L^{-1}.m^{-2}}]} & \\
        \midrule
        \multicolumn{2}{@{}l}{\textbf{Mean} $\pm$ \textbf{SD}} & & & & & & & & \\
        & Site \quad SG  & 223 \pm 190   &        & 18.0 \pm 14 & {(a)}  & 19 \pm 34 &        & 4.2 \pm 19.7 &       \\
        & \phantom{Site \quad} QdT & 175 \pm 94.5  &        & 22.8 \pm 15 & {(a)}  & 7 \pm 10  &        & 0.2 \pm 0.3  &       \\
        & \phantom{Site \quad} LC  & 234 \pm 183   &        & 22.4 \pm 13 & {(a)}  & 13 \pm 15 &        & 0.5 \pm 0.5  &       \\
        & \phantom{Site \quad} NA  & 145 \pm 169   &        & 65.0 \pm 39 & {(b)}  & 23 \pm 25 &        & 0.4 \pm 0.4  &       \\
        \addlinespace
        & Biocrust \quad BSC+ & 171 \pm 133   &        & 42.6 \pm 34 & {(b)}  & 19 \pm 22 & {(b)}  & 0.5 \pm 0.5  &       \\
        & \phantom{Biocrust \quad} BSC- & 218 \pm 190   &        & 21.5 \pm 21 & {(a)}  & 12 \pm 24 & {(a)}  & 2.1 \pm 13.7 &       \\
        \midrule
        \multicolumn{2}{@{}l}{\textbf{Site*Biocrust}} & & & & & & & & \\
        & SG \quad BSC+  & 202 \pm 103   &        & 24.8 \pm 15 & {(ab)} & 19 \pm 18 &        & 0.7 \pm 0.5  &       \\
        & SG \quad BSC-  & 244 \pm 251   &        & 11.1 \pm 10 & {(a)}  & 18 \pm 46 &        & 7.5 \pm 27.5 &       \\
        & QdT \quad BSC+ & 148 \pm 63.9  &        & 30.5 \pm 13 & {(b)}  & 9 \pm 13  &        & 0.3 \pm 0.3  &       \\
        & QdT \quad BSC- & 202 \pm 113   &        & 15.1 \pm 13 & {(ab)} & 5 \pm 6   &        & 0.2 \pm 0.2  &       \\
        & LC \quad BSC+  & 226 \pm 223   &        & 23.5 \pm 14 & {(ab)} & 17 \pm 19 &        & 0.6 \pm 0.5  &       \\
        & LC \quad BSC-  & 241 \pm 139   &        & 21.3 \pm 14 & {(ab)} & 9 \pm 9   &        & 0.4 \pm 0.3  &       \\
        & NA \quad BSC+  & 107 \pm 35.6  &        & 91.5 \pm 28 & {(c)}  & 30 \pm 32 &        & 0.4 \pm 0.4  &       \\
        & NA \quad BSC-  & 183 \pm 234   &        & 38.5 \pm 30 & {(b)}  & 16 \pm 15 &        & 0.4 \pm 0.3  &       \\
        \midrule
        \multicolumn{2}{@{}l}{\textbf{Significance} ($p$-value)} & & & & & & & & \\
        & Site         & \multicolumn{2}{c}{(ns)} & \num{1.78e-12} & \tnote{*} & \multicolumn{2}{c}{(ns)} & \multicolumn{2}{c}{(ns)} \\
        & Biocrust     & \multicolumn{2}{c}{(ns)} & \num{8.11e-08} & \tnote{*} & \num{0.00946} & \tnote{*} & \multicolumn{2}{c}{(ns)} \\
        & Site*Biocrust& \multicolumn{2}{c}{(ns)} & \num{0.00164} & \tnote{*} & \multicolumn{2}{c}{(ns)} & \multicolumn{2}{c}{(ns)} \\
        \midrule
        \multicolumn{2}{@{}l}{\textbf{Covariates}} &
        \multicolumn{2}{@{}l}{clay} &
        \multicolumn{2}{@{}l}{} &
        \multicolumn{2}{@{}l}{\makecell[tl]{clay + silt + sand +\\ C\textsubscript{T} + SOC}} &
        \multicolumn{2}{@{}l}{\makecell[tl]{clay + silt + sand\\ + C\textsubscript{T} + SOC}} \\
        \bottomrule
      \end{tabular}
      \begin{tablenotes}[para,flushleft]
        \item[a] Letters indicate significant differences between means according to post-hoc tests with Šidák correction for multiple comparisons. Levels not sharing any letter are significantly different (p < 0.05).
        \item[*] Statistically significant effect (p < 0.05).
        \item[ns] Non-significant (p >= 0.05).
        \item[SD] Standard Deviation.
        \item[C\textsubscript{T}] Total Carbon content.
        \item[N\textsubscript{T}] Total Nitrogen content.
        \item[SOC] Soil Organic Carbon.
      \end{tablenotes}
    \end{threeparttable}
  \end{table}