\chapter{Discussion}

The aim of my dissertation was twofold: (1) to explore the power of additional full-length genomes for a richer understanding of genome variation in this species, and (2) to learn how additional genomes and genome graphs, as a novel reference structure, can be leveraged to provide more value to existing short-read data sets, such as the one from the 1001 Genomes Project \citep{1001_Genomes_Consortium_Electronic_address_magnusnordborggmioeawacat2016-pn}. This required additional genome assemblies to enlarge the available sequence space as well as new methods to build genome graphs and use them. As we have discussed before, current reference based analysis is strongly influenced by the available reference genome and its quality. Missing, or misrepresented sequence causes a reference bias. This bias can be reduced by adding additional alleles to the reference genome. In an effort to add missing sequence space six \textit{de-novo} assemblies of \ath\ were created. The comparison of these highly contiguous assemblies underlined not only the high synteny of the \ath\ population, but also the power of whole genome assemblies for a more unbiased structural variation detection. A comparison of variants detected in the reference framework, using different methods showed the high variability of those calls. I deployed a new, combinatorial approach to annotate the assemblies. I created a pan-proteome based on this annotation, that showed a similar synteny. I was able to describe the changes in orthogroup sizes and discover a set of potentially ancestral genes that are missing from the current reference annotation. Using the \textit{de-novo} assemblies I then constructed a genome graph to represent the previously unknown and unrepresented sequences. I was able to show that we can already represent a large portion of the core genome of \ath, while we are still lacking representation of the shell genome. This was done by aligning a set of short-reads from 840 accessions to the complex, whole-genome derived genome graph, using a method I established. I highlighted the existing challenges of constructing a graph from multiple whole genome alignments by describing the portion of the graph not aligned to the reference. I used \textit{panSV}, a novel graph based, reference free, variant detection tool, developed for this thesis, to discover traces of the mobilome in the graph itself. In addition to enlarging the pan-genome and pan-proteome I was able to show that using a genome graph as an alignment target for reads can drastically reduce the reference induced pseudo-heterozygosity seen in short-read analyses \citep{Jaegle2021-jw}.

\section{The sixRef pan-genome}
\label{sec:theSixRefPanGenome}

The \textit{de-novo} assemblies of the six accessions are highly contiguous, to a degree where multiple chromosome arms are assembled as one contig. The chosen approach of long read sequencing, short-read polishing, with extra optical maps to ensure correct scaffolding gave me the opportunity to detect and place structural variants with a higher reliability. Despite the fact that, based on the scaffolded sequence length, we only exceeded the reference genome size in three out of six assemblies, we can still conclude, that we were able to assemble highly contiguous genomes du to the fact, that we were able to scaffold full chromosome arms without major stretches of Ns. The centromeric regions remain the problematic areas of the assemblies. Using a pairwise comparison with the reference I was not only able to show the high conservation of the \ath\ genome structure, where 77.2\% of the reference genome is syntenic with all six assembled genomes, but also observe previously described structural variants in the assemblies \citep{Zapata2016-iq,Rowan2019-ut,Jiao2020-yz}. PAVs detected from the alignments are unbiased in size. While in short-read based analysis the results were always biased towards the detection of deletions over insertions \citep{Kosugi2019-ls,Ho2020-mt}, I can now observe 0.9 Mb more inserted sequence over deleted variation. This is a direct result of the comparison of whole genome assemblies with the reference genome, as those loci missing from the reference are now assembled in the reference. In addition, the increased resolution of repeat structures easily adds to the additional sequence space that can be detected as insertions. While the current surge in telomere to telomere assemblies allows an even better representation of structural variants in genomes \citep{Gonzalez_de_la_Rosa2021-kz,Giguere2021-sy,Wang2021-vc}, the slightly incomplete six \textit{de-novo} assemblies already increase the power to detect large SVs and represent them in their sequence context. \newline
One of the accessions chosen for assembly was \textit{AT6909}. The current \textit{TAIR10} reference genome \citep{Berardini2015-is} is based on the same accession and thus our assembly has a considerably lower number of variants detected. Interestingly the distribution of detected variant sizes shifts from a majority of \textit{SNPs}, in all the other accessions, to larger variants. This is the direct result of the changes that stem from the advances in genome sequencing and assembly methods, as already described by \citep{Wang2021-vc}. In addition to the shift in variant sizes I can also observe a shift in SV type, as copy number changes are far more common in this comparison, than in all the other accessions. As the assembly is highly similar to the reference, real structural rearrangements are rare and the observed copy number changes reveal the sequence of the previously hard-to-assemble parts taht were missing in the reference genome. \newline
I had initially planned to intersect three different variant sets of the six accessions in an effort to create a reliable set of variants for method validation. This effort failed as the variations that were no SNPs were far too dissimilar. Nevertheless a valuable lesson on the method-inherent biases was learned, as variants called by different methods were hard to compare. Different approaches rely on fundamentally different concepts to detect variants. While short-read based methods are sufficient to describe small base pair differences, changes in coverage distribution, and insert sizes, they suffer from the previously described reference bias. These methods are mostly blind to the syntenic concept that can be employed by whole genome comparisons that are able to describe large-scale variation. The assembly based methods are not entirely without their issues either, as they suffer, for example, from the exclusion of contigs in the analysis, as in the \textit{SyRI} analysis, or struggle with the detection of inter-chromosomal rearrangements. In addition the classification of certain regions into HDRs leaves some regions under-resolved. The graph allows repetitive regions to collapse, similar to the short-read alignments, while maintaining the sequence context. The detection of variants can nevertheless be hindered by the inability to project complex variants into the linear reference space, or simply under-aligned regions during graph construction. The high identity of SNPs between the different call sets shows that small variation can be called reliably from every type of input data, and are more trustworthy, whereas longer variation is more unreliable in its recallability. \newline
The remaining differences in short variation can easily be explained as a result of different variant reporting. For example a short multi base pair variant called by an assembly based method may be reported as multiple SNPs in the short-read calls. This explains a large proportion of the SNPs that are not shared between all call sets and some of the small variation. Another example is the direct result of the described reference bias. As genome assemblies are contiguous sequences, resolved copy number changes result in PAV events insted of heterozygous variant calls. As a result of this small variant calls in the short-read based set can proxy for larger SVs. An advantage of assembly based variant detection methods is their ability to detect larger and more complex variation due to their longer continuity over short-reads. While the differences described above are a result of the properties of the query sequence in the comparison, another critical factor, in addition to the alignment method itself, comes from the available reference sequence data. The sequenced short-read alignments contain in principle the full range of sequence variation available but suffer from being mapped to a reference that does not contain all the sequences present in the query genome and mapping therefore misplaces short-reads, producing variant calls in inappropriate places, or lack the power to detect them entirely. In contrast the pairwise whole genome alignments use largely resolved chromosomal sequences to detect variation, but the assembly itself can lack sequence that would be represented in the short-reads. Furthermore, the tool \textit{SyRI} aligns chromosome scale scaffolds to reference chromosomes, and thus is unable to call any variation that is present in unplaced contigs. The graph that has been used in this analysis also contains the unplaced contigs and therefore is able to utilize more genomic sequence than the pairwise alignment. Irrespective of the additional availability of sequence, the variants detected from the graph strongly depend on the quality of the graph structure and the ability of the algorithm to deal with highly complex regions. This results in either a reduced number of variants, or a concatenation into larger, under aligned blocks.

\section{The sixRef proteome}
\label{sec:theSixRefProteome}

The \textit{auto-ant} annotation pipeline, which I developed for this thesis, could reliably annotate genomes. Especially the inclusion of \textit{evidenceModeler} to combine multiple independent gene predictions improved the overall results. In the settings I opted for a higher specificity over more annotated features. The \textit{augustus} annotation software has been a mainstay for ab-initio gene prediction and has been used in multiple other annotation pipelines, such as \textit{BRAKER} \citep{Hoff2019-zi}. This novel pipeline increases the annotation speed of \textit{augustus} by chunking the assemblies and enables the tool to run annotations of multiple assemblies in parallel. The annotations of different assemblies are then related to each other using an orthogroup assignment. The usefulness of the upfront TE annotation and its masking is proven by the very low number of TEs annotated in the final gene annotation, despite their presence in the genomes and the ability of \textit{augustus} to annotate these. This is especially useful to deal with retrotransposons that have an RNA intermediate \citep{Wicker2007-nv} and would otherwise be annotated by the RNA based annotation steps. The, in comparison, higher number of transcript related features in the RNA based \textit{augustus} annotation supports the decision to rate the trustworthiness of this annotation higher than the other annotations. The high similarity of the six genome annotations in raw numbers and the orthogroup assignments, despite the highly variable mRNA evidence, further supports the robust performance of the \textit{auto-ant} pipeline. Errors in the annotation mostly occur in the form of gene fusions that create overly long gene transcripts, but those events are rare. The gene annotations are coherent with the reference annotation. Most of the genes are orthologous to genes in the \textit{araport11} annotation and conserved in their order, orientation and copy number. This, again, highlights the strong synteny of the assembled genomes and reference genome. The localization analysis of genes showed the same structural variation pattern as discovered in the \textit{SyRI} analysis. The TE annotation using \textit{EDTA} also showed the expected patterns of TE class distribution. Gypsy LTR TEs are more prevalent in pericentromeric regions, while copia LTR TEs more common in gene dense parts of the genome \citep{Hufford2021-ed}. \newline
The pan-proteome follows the expected U-shaped distribution of \textit{private}, \textit{shell}, and \textit{core} orthogroups that can be observed in all pan-genomic analysis. I can even observe the reminiscence of ancient genome duplications in the duplication patten of orthogroup copy numbers \citep{Simillion2002-ub,Del_Pozo2015-id}. In the reference free analysis I observed that \textit{core} orthogroups were expanded more often than contracted. This behavior is an artifact of the \textit{core} definition. In order to be considered as a \textit{core} orthogroup, this group has to contain at least one gene from each annotation. As the majority of orthogroups are single copy orthogroups, the loss of a gene copy in a single accession would remove this orthogroup from the \textit{core} set. In contrast the reference based orthogroup description contains more contracted orthogroups. This is another example of reference bias. \newline
The \textit{araport11} reference annotation \citep{Cheng2017-ah} contains more genes than the more conservative \textit{de-novo} annotation produced by \textit{auto-ant}. As a result of this, the additional genes in the manually curated reference annotation create the erroneous impression that the orthogroups are contracted. Despite this, I have been able to discover 148 orthogroups that are shared among all assembled accessions, and the outgroup, \ara, but are not represented in the reference annotation. These genes are most likely core genes that are not represented in the reference annotation. The fact that 62.9\% of them are located inside variable regions in the graph makes it very likely that their genomic sequences are not present in the reference assembly, further adding to the potential bias of the reference. While only 30.3\% of them are supported by RNA transcription evidence, which is considerably lower than the average support of annotated genes (78.8\%), at least these have a high confidence of being expressed genes that are part of the pan-proteome of \ath, but are not represented in the known reference annotation. The remaining genes might be pseudogenes, or artifacts of the annotation approach, that happen to intersect with annotation of the outgroup, that has also been annotated using \textit{augustus}. An in-depth analysis of these genes would be interesting, but has not been performed yet. 

\section{Graph genome}
\label{sec:graphConstructionDiscussion}

The constructed genome graph is able to represent the pan-genome of the six \ath\ accessions and the \textit{TAIR10} reference genome \citep{Berardini2015-is}. The 79\% graph \textit{core} sequence in the reference genome corresponds well with the 77.2\% of syntenic sequence in the reference that is shared among all genomes. The difference can easily be explained by the slightly different definitions. The syntenic sequence detected by \textit{SyRI} contains internal variation, while the graph collapses repetitive regions. The good representation of sequence synteny in the graph is, in part, the result of the additional smoothing step in the \textit{pggb} pipeline. This step increased linearity and alignment rate in the graph, as indicated by the decreased node degree. This drop is the result of the alignment of previously unaligned sequences of the graph and the splitting of over-connected components in the graph. This can also be observed in the decreased node size. Despite the best effort to resolve the graph, overly connected nodes and misalignments still prevail in the graph. This becomes very obvious in the \textit{Kraken2} \citep{Wood2019-jh} analysis of the non-reference sequences. 79.4\% of the classified sequence was attributed to \ath. This is a direct result of unaligned sequences that exist in the graph, but are represented by the reference genome. Based on the annotation of these regions it becomes obvious that a majority of them are either repetitive or belong to the mobilome. As graph construction has to be a trade-off between compression and usability the limit that I had to impose on repeat copies to collapse, as well as the synteny driven underlying alignment will have had an impact on the presence of unaligned mobilome sequences. In turn this has resulted in a more linear graph, where the pericentromeric regions are highly connected among the chromosomes, but the chromosome arms are mostly linear. The overall compression rate of sequence in the graph is comparable for all six \textit{de-novo }assemblies. Only the \textit{TAIR10} reference genome exhibits a lower compression rate. This is most likely a result of the under-resolved repeat space in the reference that is being compressed in the \textit{de-novo} assemblies. \newline
The use of \textit{panSV} enables a better resolution of the variation in the pan-genome over traditional reference based methods. When using this method we have to keep in mind that the representation of variation by \textit{panSV} is very different compared to the traditional vcf format. This is a result of the core level based variant definition that does not require a variant to be anchored to a singular reference genome, but describes variation in the context of pan-genome frequencies. This can explain the differences in the comparison of variable regions detected by \textit{panSV} and the results of the \textit{TAIR10} \citep{Berardini2015-is} based \textit{vg deconstruct} \citep{Garrison2018-qh} results, that were both obtained from the same genome graph. Most variation reported by \textit{panSV} is present at the highest \textit{core} level, and intersects with the reference based variants. The strength of \textit{panSV} becomes evident in the complex and nested regions of the graph. While \textit{vg deconstruct} is not able to fully represent the nested variation, \textit{panSV} resolves those complex regions and reports a multitude of variants that are nested within, where \textit{vg deconstruct} only reports a single large variant. This is the reason for the higher fraction of SNPs called by \textit{panSV}, and the lower fraction of \textit{small variants}. An additional driver for the difference in detected variation are repetitive regions, due to the way they are represented in the graph. As \textit{panSV} searches for regions with diverging \textit{core} levels, it is unable to describe repeats that do not result in a change in the \textit{core} level. Nevertheless, this new approach will enable us to access nested variation that has been hard to describe and might become an additional method to describe the complex variation in a more conceivable form. \newline
Despite the shortcomings in the graph resolution I was able to detect parts of the mobilome in the variable regions of the graph using \textit{panSV}. While the basic algorithm is very simplistic in its description of the graph, the reference free approach helps to deal with variation that tools like \textit{vg deconstruct} struggle to represent. In highly repetitive regions it can be challenging to project variation into a linear coordinate system. This problem is entirely circumvented by my approach. Nevertheless it does not come without its own set of challenges. One of which is the interpretation of the results. While the concept of polarized, reference based variation is well understood and easy to grasp, the complexity of traversals and their nestedness can be hard to understand and even harder to visualize. This concept of pan-genomic variation will need time to be refined and get used to. Nonetheless I was able to successfully use this approach to detect the traces of insertion mechanisms of the mobilome and explain a subset of the non-standard orthogroups. TEs, genes and a combination of both is enriched in variable regions that are anchored by identical, thus repeated, nodes on both ends. This is a result of the insertion and breakpoint repair mechanism \citep{Chatterjee2017-dk}. The high fraction of those regions that contain both, TEs and non-standard orthogroup genes, indicates that those genes might be dragged through the genome alongside TE.s A more in-depth analysis of the remaining bubbles could possibly reveal previously undescribed players of the mobilome. 

\section{Graph based short-read alignments}
\label{sec:graphBasedShortReadAlignments}

Alignments to graphs are one of the biggest challenges when working with genome graphs. Only a limited number of genome graph alignment algorithms habve been implemented at the moment, and most of them have been geared towards vcf-derived \textit{vg} graphs. For this project I evaluated the performance of different algorithms on complex \textit{pggb} graph \citep{Garrison2023-lj}. Two of the alignment algorithms are part of the \textit{vg} toolkit (\textit{vg map}, and \textit{vg giraffe}) \citep{Garrison2018-qh}. I also used \textit{graphAligner} \citep{Rautiainen2019-wj}, and a novel combination of \textit{bwa mem} \citep{Li2013-kr} and \textit{vg inject} to project linear alignments into the graph space. While \textit{vg map}, and \textit{vg giraffe} had a superior performance on linear, and vcf-derived graphs their memory consumption increased, and alignability decreased on complex graphs. Here they struggle with the highly complex regions of the graph that challenge their alignment algorithm and bloat the calculations they need to perform to a degree where alignments become infeasible. While \textit{graphAligner} did not suffer from the same limitations, and was exceptionally fast, its implementation for long-reads made it unsuitable for short-read alignments and resulted in an inflated amount of covered sequence. A short-read implementation of the algorithm is in development, but has not been released at this point. This left me with the \textit{vg inject} based approach, that by design had a constant performance on all suitable graphs, but suffered from its own set of limitations. The underlying idea has a very simplistic beauty in that it uses the well established alignment method to flat sequences and the positional relationship between flat genomes and paths in the graph to then inject the alignments into the graph. While the additional sequences present in the assembled genomes allow more reads to be aligned, the fact that the initial alignment target contains multiple copies of the same sequence unnecessarily slows down the alignment step. This can become infeasible with more, and larger genomes. In addition this method can only align reads to allele combinations represented in one of the genomes and crossovers are not possible. Still it is the best option to date to align short-reads to highly complex graphs. \newline
In addition to the evaluation of alignment tools we can also observe the impact of further graph compression onto the alignment statistics. While the fraction of reads aligned by \textit{vg giraffe} stays almost identical for the step from the \textit{chromosome graph} to the \textit{linear graph}, the amount of covered sequence decreases by almost 2 Mb when aligning to the \textit{complex graph}. This is a result of the increased compression, especially in the pericentromeric regions. \newline
The alignment of short-read sets from 840 accessions from the 1001 Genomes Project \citep{1001_Genomes_Consortium_Electronic_address_magnusnordborggmioeawacat2016-pn} demonstrates that the graph makes additional sequence available as additional mapping targets increase the number of alignable reads. This is directly reflected in the ability to cover more sequence in the graph than the length of the \textit{TAIR10} reference genome, and align more reads compared to alignments to the \textit{TAIR10} reference genome. This is especially visible in re-mapping of the three accessions that are part of the graph, and the 1001 Genomes Set. The amount of sequence covered by these alignments exceeds the length of their repective genome assembly. This is result of sequence that could not be assembled in their genome assembly, is reperesented by one of the other assemblies and thus becomes availabel for mapping. This demonstartes that multiple incomplete references in a graph can together represent more of their individual sequences. The ability to cover more sequence in the graph than the length of the reference genome is also a true biological signal of additional alleles from the novel assemblies, as well as a better representation of extensive copy number variations present in \ath\ \citep{Jaegle2021-jw}. Nevertheless, a varying fraction of reads remained unaligned in each of the short-read sets. Only a small fraction of these reads could be assigned to \textit{viridiplantae}. Instead a larger fraction has been classified as belonging to a \textit{non-viridiplantea} clade and as such are most likely a result of samples not coming from sterily grown plants, which are naturally colonized by microbes. As \textit{Kraken2} \citep{Wood2019-jh} masks repetitive regions, the majority of unclassified reads most likely also originate from those. This is in line with the observation that the amount of unaligned genome sequence, by the k-mer based size estimation, is largely independent from the amount of unaligned reads. Nevertheless the sequences in the contaminated sets slightly bias genome size estimation analysis. \newline
The analysis of the aligned, and estimated genome size reveals another important bias in science. While there is a significant correlation between the genome size and the admixture group, there is an even stronger correlation with the laboratory that sequenced the data set. Especially the Swedish accessions, which show the highest amount of covered sequence were mostly sequenced at a single sequencing center that sequenced few other accessions in the 1001 Genomes Project. It is very likely that the size differences are a bias that was introduced by different handling of the plants and material, or sequencing protocols by different experimenters \citep{Stoler2021-ea}. \newline
I already described the pan-genome of the six \textit{de-novo} genome assemblies. By using the graph as a target for short-read alignment I can now use it to describe the pan-genome of the mapping population. A shift towards the higher pan-genome levels can be observed in the comparison of the two pan-genomes. Especially the shift from \textit{private} sequence to \textit{shell} sequence is noticeable. This shift is a result of the glass roof imposed by the graph, and another form of reference bias. As the sequence space in the graph is finite I can only describe sequences that are present in the graph, and thus the identification of novel sequences is impossible. This drives the shift away from \textit{private} sequences in the mapping population as previously \textit{private} sequences are in fact underrepresented in the assemblies. Another driver is the under alignment of the graph, that I described in the non-reference analysis. As this sequence is in fact more common than it is perceived in the graph based pan-genome it becomes a \textit{shell}, or even a \textit{core} sequence in the mapping based pan-genome. Nevertheless the fact that 82.2\% of the core sequence is identical in both pan-genomes shows that we can already represent a large fraction of the \ath\ \textit{core} genome with just seven, well chosen, assemblies. \newline
In the pan-proteome the same behavior as in the pan-genome analysis can be observed. Again the number of \textit{core} orthogroups increases while the \textit{shell}, and especially \textit{private} categories shrink. The shift in \textit{shell}, and \textit{private} graph sequence to the next (higher) category is a hint at underestimated alleles that appear to be rare in the graph, but in fact are more common in the larger population. This theory is backed by the per-accession analysis of the pan-proteome. The three accessions that are part of both, the assembled genomes, and the mapping population, have higher numbers of \textit{private} orthogroups in the mapping population based analysis, compared to the other accessions in the mapping population. Especially the high number of \textit{private} orthogroups in short-read mapping of the relict accession \textit{AT6911} shows that we are still missing rare alleles of the wider population. Adding more diverse genomes to the graph structure will open up additional alleles. The pan-proteome analysis revealed another way the current \textit{TAIR10} reference genome imposes a bias onto analysis performed with it. The set of 148 \textit{core} orthogroups that contain a member form \ara\, but have no member from the reference annotation, are also present in the majority of the mapping population, further underlining their status as common genes of \ath. While this problem is not as severe as for example in maize \citep{Hirsch2014-dm,Lu2015-mm,Hirsch2016-je,Jin2016-cd}, or in wheat \citep{Bayer2022-lg}, researchers in \ath\ have already begun to choose alternative references to better represent the causal genetic components for their research question \citep{Wojtowicz2021-eq}. Here, genome graphs and pan-genomes can help us to better represent and understand the true genomic potential of a species. While we are already capable of expanding the core genome of \ath\ with the addition of just six assemblies, it also means that we will need to add more diverse accessions to be able to represent the rare alleles of the population. Never the less the finite sequence space of a graph will never be able to represent the full library of sequences. \newline
Variant detection is the kind of reference-based analysis that suffers the most from reference bias. Therefore the analysis of the graph based calls can help us to better understand how a genome graph changes, and reduces the reference bias in such analysis. Similar to the analysis of the variation stored in the graph itself, the distribution of variant types shifts from predominantly \textit{SNPs}, and a few rare \textit{small variants}, to an almost equal number of sites that are categorized as \textit{SNPs} and \textit{large variants}. The underrepresentation of larger variants in previous analysis is a result of their inability to detect them. In a graph these variants are represented and can be called. Such variants are either a representation of large non-reference alleles, or of an incomplete graph resolution. This is further supported by the localization of \textit{large variants} in the genome. While \textit{SNPs} are distributed along the chromosome, \textit{small} and \textit{large variants} are mostly found in the pericentromeric regions, where the \textit{de-novo} assemblies were able to sequence more, and deeper into complex regions of the genome. The main difference between \textit{SNPs} and \textit{large variants} is their frequency within the mapping population. Sites that are classified as \textit{SNPs} are shared by more samples, while \textit{large variants} are mostly private to a single sample. Even though the sequence of the underlying PAV event might not be rare in the population, nested variation within these regions result in a multitude of low frequency variants that differ by only a few bases, and thus seem unique. This also artificially inflates the heterozygosity rate in the population. The potential causes of this kind of new reference bias will be discussed at the end of this section. \newline
The comparison with the calls made by the 1001 Genomes Project \citep{1001_Genomes_Consortium_Electronic_address_magnusnordborggmioeawacat2016-pn}, using the same short-read data reveals a reduction and shift in reference bias. While the number of variants detected by both methods are very similar, they only intersect for 70.1\% of the sites per accession. This can partly be attributed to the fact that the graph allows to call larger variants, but also the difference in the post processing of the variant calls. The graph calls were subjected to a very basic quality filtering that kept most of the variants, while the 1001 Genomes Project calls have been extensively filtered and heterozygous calls have mostly been removed. This means that in reality the graph produced substantially fewer variant calls. This can also be seen in comparison with the re-analysis of the reads in the heterozygosity study by Jaegle et. al. \citep{Jaegle2021-jw}. Their initial calls resulted in 3.3 million SNPs. Which is almost twice the amount of total sites I called from the graph for all variant types, and three times the number of SNPs. Even though the majority of variants, called by the 1001 Genomes Project, intersect, or overlap with the variants called from the graph, not all variants could be recalled. Even though no variants were called at these positions, 90.9\% of them were covered by reads in the graph. This means the reads aligned over the previously variable position matched perfectly. As such the previous variant was probably a result of the inclomplete sequence representation in the reference genome. \newline
Multiple factors are responsible for the differences between the two call sets. They are either true biological signals as a result of the reduced reference bias, or a new bias that has been introduced by the graph and the way it has been constructed. First of all the better representation of the pan-genome sequence in the graph reduces the number of misplaced alignments that would otherwise result in variant calls caused by the incomplete representation of the reference genome. In addition small variants that have been called in the 1001 Genomes Project may actually proxy for larger variants, which could not be represented by the short-reads and the reference sequence, but are now resolved, and represented by the \textit{de-novo} assemblies. The covered variants that could not be recalled probably belong to this category. In addition the graph itself also biases the variant calls. This bias, and its result will be discussed at the end of this section. \newline
Beyond the simple comparison of intersecting variants, the rate of heterozygosity also tells an interesting story. As described before, the main contributors to heterozygosity in \ath\ are extensive gene duplications that are not represented in the linear reference genome \citep{Jaegle2021-jw}. The genome graph enables us to better represent this set of variable sequences and therefore correctly place the corresponding reads. This results in an overall reduction of SNPs and heterozygous calls. I call just 44.4\% of the SNPs have been called by Jaegle et.al. and only 4.6\% of them were heteroygous per line, which is a massive reduction. Some of this is due to the fact that I used 840 accessions, instead of 1,057, but the main contributor to this reduction is the better sequence representation in the graph. Nevertheless the overall heterozygosity (14.2\%) in the variants remains higher than expected in a selfing plant. The cause for this can be twofold. As we can observe in the distribution of heterozygous calls in \textit{AT6909}, most of them are located around the pericentromeric regions that are expected to have a higher mutation rate and therefore maintian more variants eben in selfing plants. Thus some of them are real heterozygous calls, but a second group are them ares probably the result of a new type of reference bias that is introduced by the graph representation of the sequence, which I will discuss next. \newline
While the graph resolves some problems surrounding the incomplete sequence representation of the reference genome, it also introduces a new type of bias. In the graph, sequences with variable copy numbers are either compressed into subgraphs where the differences between copies are collapsed, or left unaligned as large PAVs in the graph. As a result of this we can observe a higher number of \textit{small} and \textit{large variants} in the variant calls. They either represent the under aligned fraction of the graph, or genuine, new sequences. In addition this also results in a inflation of heterozygous calls. While the differnt copies of a region are assembled and represented in the graph, collapsed sub graphs can make the placement of novel variation dificult and result in heterozygous calls in these sub graphs. In addition nested varition in larger variants also creates heterozygous calls, as they can not be represented in the reference based vcf file. This problem is especially prevalent in the mobileome, and can be observed in the comparison of mappings of the reference accession, \textit{AT6909} with the \textit{TAIR10} reference genome. It is clearly visible that the heterozygous variants coincide with regions of high TE density. As the pericentromeric regions have not been fully resolved the diverging TE copies easily cause heterozygous variant calls. The subset of heterozygous calls with two alternative alleles in this accession highlights that we have not fully resolved it in the \textit{de-novo} assemblies. The exact degree by which this bias influences the variants calls needs to be determined, and addressed in the future.


